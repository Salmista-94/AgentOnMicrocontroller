\rhead{\bf mc\_AclSend()}
\noindent
\vspace{5pt}
\rule{6.5in}{0.015in}
\noindent
\phantomsection
{\LARGE \bf mc\_AclSend()\index{mc\_AclSend()}}\\
\addcontentsline{toc}{section}{mc\_AclSend()}
\label{api:mc_AclSend()}

\noindent
{\bf Synopsis}\\
{\bf \#include $<$libmc.h$>$}\\
{\bf int mc\_AclSend}({\bf MCAgency\_t} $attr$, {\bf fipa\_acl\_message\_t*} $acl$);\\

\noindent
{\bf Purpose}\\
Send an ACL message.\\

\noindent
{\bf Return Value}\\
Returns 0 on success, non-zero on failure.\\

\noindent
{\bf Parameters}
\vspace{-0.1in}
\begin{description}
\item
\begin{tabular}{p{10 mm}p{145 mm}} 
$attr$ & An initialized Mobile-C agency handle.\\
$message$ & The ACL message to send. 
\end{tabular}
\end{description}

\noindent
{\bf Description}\\
This function will compose a fully compliant FIPA Acl message
and send it to the destinations as specified by the 'receiver' field of
the acl message. The function also creates a FIPA compliant
xml envelope which is attached to the message. The message is sent
using the FIPA compliant HTTP Message Transport Protocol.\\

\noindent
{\bf Example}\\
\noindent
{\footnotesize \verbatiminput{../demos/FIPA_compliant_ACL_messages/fipa_test/test2.xml}}

\noindent
{\bf See Also}\\
\texttt{
  mc\_AclNew(), mc\_AclPost(), mc\_AclReply(), mc\_AclRetrieve(), 
    \linebreak mc\_AclWaitRetrieve()
}

%\CPlot::\DataThreeD(), \CPlot::\DataFile(), \CPlot::\Plotting(), \plotxy().\\
