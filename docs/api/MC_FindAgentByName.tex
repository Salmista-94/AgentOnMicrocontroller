\rhead{\bf MC\_FindAgentByName()}
\noindent
\vspace{5pt}
\rule{6.5in}{0.015in}
\noindent
{\LARGE \bf MC\_FindAgentByName()\index{MC\_FindAgentByName()}}\\
\phantomsection
\addcontentsline{toc}{section}{MC\_FindAgentByName()}

\noindent
{\bf Synopsis}\\
{\bf \#include $<$libmc.h$>$}\\
{\bf MCAgent\_t MC\_FindAgentByName}({\bf MCAgency\_t} $agency$, {\bf const char *}$name$);\\

\noindent
{\bf Purpose}\\
Find a mobile agent by its name in an agency.\\

\noindent
{\bf Return Value}\\
The function returns an {\bf MCAgent\_t} object on success or NULL on failure.\\

\noindent
{\bf Parameters}
\vspace{-0.1in}
\begin{description}
\item               
\begin{tabular}{p{10 mm}p{145 mm}}
$agency$ & An agency handle.\\
$name$ & A character string containing the mobile agent's name.
\end{tabular}
\end{description}

\noindent
{\bf Description}\\
This function is used to find and retrieve a pointer to an existing running 
mobile agent in an agency by the mobile agent's given name.\\

\noindent
{\bf Example}\\
\noindent
{\footnotesize\verbatiminput{../demos/cspace-agentspace_interface/persistent_example/server.c}}

\noindent
{\bf See Also}\\
MC\_FindAgentByID()

%\CPlot::\DataThreeD(), \CPlot::\DataFile(), \CPlot::\Plotting(), \plotxy().\\
