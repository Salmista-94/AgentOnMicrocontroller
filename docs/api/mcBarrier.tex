\rhead{\bf mc\_Barrier()}
\noindent
\vspace{5pt}
\rule{6.5in}{0.015in}
\noindent
\phantomsection
{\LARGE \bf mc\_Barrier()\index{mc\_Barrier()}}\\
\addcontentsline{toc}{section}{mc\_Barrier()}
\label{api:mc_Barrier()}

\noindent
{\bf Synopsis}\\
{\bf int mc\_Barrier}({\bf int} $id$);\\

\noindent
{\bf Purpose}\\
This function blocks the calling thread until all registered threads and
agents have been blocked. \\

\noindent
{\bf Return Value}\\
This function returns 0 on success, or non-zero if the id could not be found. \\

\noindent
{\bf Parameters}
\vspace{-0.1pt}
\begin{description}
\item
\begin{tabular}{p{10 mm}p{145 mm}} 
$id$ & The id of the barrier to wait on. 
\end{tabular}
\end{description}

\noindent
{\bf Description}\\
This function is used to synchronize a number of agents and threads. Each barrier
is initialized so that it will block the execution of threads and agents until 
a predetermined number of threads or agents have activated the barrier, at which
point all blocked threads and agents will be released simultaneously. 
    \\

\noindent
{\bf Example}\\
Please see the example located at the directory
\texttt{ mobilec/demos/mc\_barrier\_example/ }. \\

\noindent

\noindent
{\bf See Also}\\
mc\_BarrierDelete(), mc\_BarrierInit(). \\

%\CPlot::\DataThreeD(), \CPlot::\DataFile(), \CPlot::\Plotting(), \plotxy().\\
