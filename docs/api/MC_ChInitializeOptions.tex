\rhead{\bf MC\_ChInitializeOptions()}
\noindent
\vspace{5pt}
\rule{6.5in}{0.015in}
\noindent
{\LARGE \bf MC\_ChInitializeOptions()\index{MC\_ChInitializeOptions()}}\\
\phantomsection
\addcontentsline{toc}{section}{MC\_ChInitializeOptions()}

\noindent
{\bf Synopsis}\\
{\bf \#include $<$libmc.h$>$}\\
{\bf int MC\_ChInitializeOptions}({\bf MCAgency\_t} $agency$, {\bf ChOptions\_t} *$options$);\\

\noindent
{\bf Purpose}\\
Set the initialization options for a Ch to be used as one AEE in an agency.\\

\noindent
{\bf Return Value}\\
The function returns 0 on success and non-zero otherwise.\\

\noindent
{\bf Parameters}
\vspace{-0.1in}
\begin{description}
\item
\begin{tabular}{p{10 mm}p{145 mm}}
$agency$ & A Mobile-C Agency.\\
$options$ & Options for setting a Ch to be used as one AEE in an agency. 
{\bf ChOptions\_t} is defined as a structure as the following:
\verbatiminput{api/ChOptions.txt}
\end{tabular}
\end{description}

\noindent
{\bf Description}\\
This function sets up a Ch for executing the mobile agent code. 
The Ch shell type and the startup file to be used are indicated in the 
argument $options$. 
If this function is not called, the default value for ChOptions will be used 
to start up a Ch for running the mobile agent code.\\

\noindent
{\bf Example}\\
\noindent
{\footnotesize\verbatiminput{../demos/miscellaneous/mc_sample_app/server.c}}

\begin{comment}
\begin{verbatim}
MCAgency_t agency;
ChOptions_t ch_options;
ch_options.chhome = malloc(50);
strcpy(ch_options.chhome, "/home/user/");
agency = MC_Initialize(5050, NULL);
MC_ChInitializeOptions(agency, ch_options);

/* Etc... */
\end{verbatim}
\end{comment}
\noindent

\noindent
{\bf See Also}\\

%\CPlot::\DataThreeD(), \CPlot::\DataFile(), \CPlot::\Plotting(), \plotxy().\\
