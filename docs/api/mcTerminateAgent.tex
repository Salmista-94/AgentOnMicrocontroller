\rhead{\bf mc\_TerminateAgent()}
\noindent
\vspace{5pt}
\rule{6.5in}{0.015in}
\noindent
{\LARGE \bf mc\_TerminateAgent()\index{mc\_TerminateAgent()}}\\
\phantomsection
\addcontentsline{toc}{section}{mc\_TerminateAgent()}
\label{api:mc_TerminateAgent()}

\noindent
{\bf Synopsis}\\
%{\bf \#include $<$mobilec.h$>$}\\
{\bf int mc\_TerminateAgent}({\bf const char*} $agent\_name$);\\

\noindent
{\bf Purpose}\\
Terminate the execution of a mobile agent in an agency.\\

\noindent
{\bf Return Value}\\
The function returns 0 on success and an error code on failure.\\

\noindent
{\bf Parameters}
\vspace{-0.1in}
\begin{description}
\item
\begin{tabular}{p{20 mm}p{145 mm}}
$agent\_name$ & The name of a valid mobile agent.
\end{tabular}
\end{description}

\noindent
{\bf Description}\\
This function halts a running mobile agent. 
The Ch interpreter is left intact. 
The mobile agent may still reside in the agency in MC\_AGENT\_NEUTRAL mode if 
the mobile agent is tagged as 'persistent', or is terminated and flushed 
otherwise.\\

\noindent
{\bf Example}\\
\noindent
{\footnotesize\verbatiminput{../demos/cspace-agentspace_interface/persistent_example/test3.xml}}

\noindent
{\bf See Also}\\

%\CPlot::\DataThreeD(), \CPlot::\DataFile(), \CPlot::\Plotting(), \plotxy().\\
