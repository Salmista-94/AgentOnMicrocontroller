\rhead{\bf MC\_DeleteAgent()}
\noindent
\vspace{5pt}
\rule{6.5in}{0.015in}
\noindent
\phantomsection
{\LARGE \bf MC\_DeleteAgent()\index{MC\_DeleteAgent()}}\\
\addcontentsline{toc}{section}{MC\_DeleteAgent()}
\label{api:MC_DeleteAgent()}

\noindent
{\bf Synopsis}\\
{\bf \#include $<$libmc.h$>$}\\
{\bf int MC\_DeleteAgent}({\bf MCAgent\_t} $agent$);\\

\noindent
{\bf Purpose}\\
Delete a mobile agent from an agency.\\

\noindent
{\bf Return Value}\\
The function returns 0 on success and non-zero otherwise.\\

\noindent
{\bf Parameters}
\vspace{-0.1in}
\begin{description}
\item
\begin{tabular}{p{10 mm}p{145 mm}} 
$agent$ & An initialized mobile agent.
\end{tabular}
\end{description}

\noindent
{\bf Description}\\
This function halts and marks an agent for removal from an agency. This 
function completely eliminates the agent, even if the agent has remaining
unfinished tasks. \\

\noindent
{\bf Example}\\
\noindent
%FIXME: Need an example here
%{\footnotesize\verbatiminput{../demos/multiple_agency_example/server.c}}

\noindent
{\bf See Also}\\
MC\_AddAgent()

%\CPlot::\DataThreeD(), \CPlot::\DataFile(), \CPlot::\Plotting(), \plotxy().\\
