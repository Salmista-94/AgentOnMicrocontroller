\rhead{\bf MC\_AclGetConversationID()}
\noindent
\vspace{5pt}
\rule{6.5in}{0.015in}
\noindent
\phantomsection
{\LARGE \bf MC\_AclGetConversationID()\index{MC\_AclGetConversationID()}}\\
\addcontentsline{toc}{section}{MC\_AclGetConversationID()}
\label{api:MC_Acl_GetConversationID()}

\noindent
{\bf Synopsis}\\
{\bf \#include $<$libmc.h$>$}\\
{\bf const char* MC\_AclGetConversationID}({\bf fipa\_acl\_message\_t*} acl);\\

\noindent
{\bf Purpose}\\
Get the conversation id of an ACL message.\\

\noindent
{\bf Return Value}\\
Returns a character string on success of NULL on failure.\\

\noindent
{\bf Parameters}
\vspace{-0.1in}
\begin{description}
\item
\begin{tabular}{p{10 mm}p{145 mm}} 
$acl$ & An initialized ACL message. 
\end{tabular}
\end{description}

\noindent
{\bf Description}\\
This function gets the ``conversation-id'' field from an ACL message. 
The conversation ID is used to differentiate multiple agent conversations which
may be happening simultaneously between two agents. For more details, please
consult the FIPA specifications at \texttt{http://www.fipa.org}.\\

\noindent
{\bf See Also}\\
\texttt{
  MC\_AclSetPerformative(), MC\_AclSetSender(), MC\_AclAddReceiver(), 
    \linebreak MC\_AclAddReplyTo()
}

%\CPlot::\DataThreeD(), \CPlot::\DataFile(), \CPlot::\Plotting(), \plotxy().\\
