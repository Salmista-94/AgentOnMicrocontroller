\rhead{\bf MC\_CondReset()}
\noindent
\vspace{5pt}
\rule{6.5in}{0.015in}
\noindent
\phantomsection
{\LARGE \bf MC\_CondReset()\index{MC\_CondReset()}}\\
\addcontentsline{toc}{section}{MC\_CondReset()}

\noindent
{\bf Synopsis}\\
{\bf \#include $<$libmc.h$>$}\\
{\bf int MC\_CondReset}({\bf MCAgency\_t} $agency$, {\bf int} $id$);\\

\noindent
{\bf Purpose}\\
Reset's a Mobile-C condition variable so that it may be used with MC\_CondWait() again. \\

\noindent
{\bf Return Value}\\
This function returns 0 upon success or non-zero if the condition 
variable was not found. \\

\noindent
{\bf Parameters}
\vspace{-0.1pt}
\begin{description}
\item
\begin{tabular}{p{10 mm}p{145 mm}} 
$agency$ & A Mobile-C agency. \\
$id$ & The id of the condition variable to signal. 
\end{tabular}
\end{description}

\noindent
{\bf Description}\\
This function reset's a Mobile-C condition variable, setting it back to 
unsignalled status. 

\noindent
{\bf Example}\\
Please see Program \vref{prog:binary_cond_example_server} in 
Chapter \ref{chap:synchronization}.\\

\noindent

\noindent
{\bf See Also}\\
MC\_CondDelete(), MC\_CondInit(), MC\_CondSignal(), 
  \linebreak MC\_CondReset().\\

%\CPlot::\DataThreeD(), \CPlot::\DataFile(), \CPlot::\Plotting(), \plotxy().\\
