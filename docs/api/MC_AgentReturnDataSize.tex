\rhead{\bf MC\_AgentReturnDataSize()}
\noindent
\vspace{5pt}
\rule{6.5in}{0.015in}
\noindent
\phantomsection
{\LARGE \bf MC\_AgentReturnDataSize()\index{MC\_AgentReturnDataSize()}}\\
\addcontentsline{toc}{section}{MC\_AgentReturnDataSize()}
\label{api:MC_AgentReturnDataSize()}

\noindent
{\bf Synopsis}\\
{\bf \#include $<$libmc.h$>$}\\
{\bf size\_t MC\_AgentReturnDataSize}({\bf MCAgent\_t} agent, {\bf int} task\_num);\\

\noindent
{\bf Purpose}\\
Get the size of the datatype of an array returned by a returning agent.\\

\noindent
{\bf Return Value}\\
Returns a positive value of -1 on failure.\\

\noindent
{\bf Parameters}
\begin{itemize}
\item \texttt{agent} : A return agent.
\item \texttt{task\_num} : This variable chooses which task within an agent to
retrieve the datasize.
\end{itemize}


\noindent
{\bf Description}\\
This function retrieves the size of the datatype of an array. In other words,
it is the size in bytes of a single element of the array.

\noindent
{\bf Example}\\
\noindent
Please see the example for \texttt{MC\_AgentReturnArrayDim()} on page \pageref{api:MC_AgentReturnArrayDim()}.

\noindent
{\bf See Also}\\
\texttt{
  MC\_AgentReturnArrayDim(), MC\_AgentReturnArrayExtent(), MC\_AgentReturnArrayNum(),
  MC\_AgentReturnDataGetSymbolAddr(), MC\_AgentReturnDataType(),
}

%\CPlot::\DataThreeD(), \CPlot::\DataFile(), \CPlot::\Plotting(), \plotxy().\\
