\rhead{\bf MC\_AgentReturnIsArray()}
\noindent
\vspace{5pt}
\rule{6.5in}{0.015in}
\noindent
\phantomsection
{\LARGE \bf MC\_AgentReturnIsArray()\index{MC\_AgentReturnIsArray()}}\\
\addcontentsline{toc}{section}{MC\_AgentReturnIsArray()}
\label{api:MC_AgentReturnIsArray()}

\noindent
{\bf Synopsis}\\
{\bf \#include $<$libmc.h$>$}\\
{\bf int MC\_AgentReturnIsArray}({\bf MCAgent\_t} agent, {\bf int} task\_num);\\

\noindent
{\bf Purpose}\\
Determine whether an agent's return value is an array or not.\\

\noindent
{\bf Return Value}\\
Returns 1 if it is an array, 0 if it is not array, or -1 on failure, such as if
there is no return data..\\

\noindent
{\bf Parameters}
\begin{itemize}
\item \texttt{agent} : A return agent.
\item \texttt{task\_num} : A task number.
\end{itemize}


\noindent
{\bf Description}\\
This function is used to determine if a return variable is an array or just a
scalar value.

\noindent
{\bf Example}\\
\noindent
Please see the example for \texttt{MC\_AgentReturnArrayDim()} on page \pageref{api:MC_AgentReturnArrayDim()}.

\noindent
{\bf See Also}\\
\texttt{
  MC\_AgentReturnArrayExtent(), MC\_AgentReturnArrayNum()
}

%\CPlot::\DataThreeD(), \CPlot::\DataFile(), \CPlot::\Plotting(), \plotxy().\\
