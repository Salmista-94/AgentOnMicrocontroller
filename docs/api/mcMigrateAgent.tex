\rhead{\bf mc\_MigrateAgent()}
\noindent
\vspace{5pt}
\rule{6.5in}{0.015in}
\noindent
\phantomsection
{\LARGE \bf mc\_MigrateAgent()\index{mc\_MigrateAgent()}}\\
\addcontentsline{toc}{section}{mc\_MigrateAgent()}
\label{api:mc_MigrateAgent()}

\noindent
{\bf Synopsis}\\
%{\bf \#include $<$libmc.h$>$}\\
{\bf int mc\_MigrateAgent}({\bf MCAgent\_t} $agent$, {\bf const char*} $hostname$, {\bf int} $port$);\\

\noindent
{\bf Purpose}\\
Instructs an agent to migrate to another host.\\

\noindent
{\bf Return Value}\\
The function returns 0 on success and non-zero otherwise.\\

\noindent
{\bf Parameters}
\vspace{-0.1in}
\begin{description}
\item
\begin{tabular}{p{15 mm}p{145 mm}} 
$agent$ & An initialized mobile agent. Typically, when invoked from agent space, this
  argument will be ``\texttt{mc\_current\_agent}'', which is the agent's pointer to itself.\\
$hostname$ & The new host to migrate to. \\
$port$ & The port on the new host to migrate to. \\
\end{tabular}
\end{description}

\noindent
{\bf Description}\\
This function instructs an agent to migrate to a new host. The task of the
agent is not incremented. The agent will executed whatever task it was
currently on when this function was invoked on the new host. Note that this
function only prepends a task to the agents task list. The agent still needs
to finish before the migration step occurs. \\

\noindent
{\bf Example}\\
\noindent
{\footnotesize\verbatiminput{../demos/agent_space_functionality/agent_self_migrate/test1.xml}}

\noindent
{\bf See Also}\\
MC\_MigrateAgent()

%\CPlot::\DataThreeD(), \CPlot::\DataFile(), \CPlot::\Plotting(), \plotxy().\\
