\rhead{\bf MC\_AclAddReplyTo()}
\noindent
\vspace{5pt}
\rule{6.5in}{0.015in}
\noindent
\phantomsection
{\LARGE \bf MC\_AclAddReplyTo()\index{MC\_AclAddReplyTo()}}\\
\addcontentsline{toc}{section}{MC\_AclAddReplyTo()}
\label{api:MC_Acl_AddReplyTo()}

\noindent
{\bf Synopsis}\\
{\bf \#include $<$libmc.h$>$}\\
{\bf int MC\_AclAddReplyTo}({\bf fipa\_acl\_message\_t*} acl, {\bf const char*} name, {\bf const char*} address );\\

\noindent
{\bf Purpose}\\
Add a reply-to address to the ACL message.\\

\noindent
{\bf Return Value}\\
Returns 0 on success or non-zero on failure. \\

\noindent
{\bf Parameters}
\vspace{-0.1in}
\begin{description}
\item
\begin{tabular}{p{10 mm}p{145 mm}} 
$acl$ & An initialized ACL message. \\
$name$ & Sets the name of the reply-to destination. \\
$address$ & Sets the address of the reply-to destination. 
\end{tabular}
\end{description}

\noindent
{\bf Description}\\
This function is used to add a reply-to address to an ACL message. This function 
may be called multiple times on an ACL message. each time this function is 
called, a new reply-to address is appended to the list of intended reply-to
addresses for the ACL message. \\

\noindent
{\bf Example}\\
\noindent
{\footnotesize\verbatiminput{../demos/FIPA_compliant_ACL_messages/fipa_test/test2.xml}}

\noindent
{\bf See Also}\\
\texttt{
  MC\_AclSetPerformative(), MC\_AclSetSender(), MC\_AclAddReceiver(), \linebreak
    MC\_AclSetContent()
}

%\CPlot::\DataThreeD(), \CPlot::\DataFile(), \CPlot::\Plotting(), \plotxy().\\
