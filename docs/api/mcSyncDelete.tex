\rhead{\bf mc\_SyncDelete()}
\noindent
\vspace{5pt}
\rule{6.5in}{0.015in}
\noindent
\phantomsection
{\LARGE \bf mc\_SyncDelete()\index{mc\_SyncDelete()}}\\
\addcontentsline{toc}{section}{mc\_SyncDelete()}

\noindent
{\bf Synopsis}\\
%{\bf \#include $<$mobilec.h$>$}\\
{\bf int mc\_SyncDelete}({\bf int} $id$);\\

\noindent
{\bf Purpose}\\
Delete a previously initialized synchronization variable.\\

\noindent
{\bf Return Value}\\
This function returns 0 on success and nonzero otherwise.\\

\noindent
{\bf Parameters}
\vspace{-0.1in}
\begin{description}
\item
\begin{tabular}{p{10 mm}p{145 mm}}
$id$ & The id of the condition variable to delete.
\end{tabular}
\end{description}

\noindent
{\bf Description}\\
This function is used to delete and deallocate a previously initialized 
Mobile-C synchronization variable.\\
 
\noindent
{\bf Example}\\
Please see the example for MC\_SyncDelete() on page
\pageref{api:MC_SyncDelete()}.\\
\noindent
%Compare with output for examples in \CPlot::\Arrow(), \CPlot::\AutoScale(),
%\CPlot::\DisplayTime(), \CPlot::\Label(), \CPlot::\TicsLabel(), 
%\CPlot::\Margins(), \CPlot::\BoundingBoxOffsets(), \CPlot::\TicsDirection(),\linebreak
%\CPlot::\TicsFormat(), and \CPlot::\Title().
%{\footnotesize\verbatiminput{template/example/Data2D.ch}}

\noindent
{\bf See Also}\\
mc\_SyncInit().\\

%\CPlot::\DataThreeD(), \CPlot::\DataFile(), \CPlot::\Plotting(), \plotxy().\\
