\rhead{\bf mc\_PrintAgentCode()}
\noindent
\vspace{5pt}
\rule{6.5in}{0.015in}
\noindent
{\LARGE \bf mc\_PrintAgentCode()\index{mc\_PrintAgentCode()}}\\
\phantomsection
\addcontentsline{toc}{section}{mc\_PrintAgentCode()}

\noindent
{\bf Synopsis}\\
%{\bf \#include $<$mobilec.h$>$}\\
{\bf int mc\_PrintAgentCode}({\bf MCAgent\_t} $agent$);\\

\noindent
{\bf Purpose}\\
Print a mobile agent code for inspection.\\

\noindent
{\bf Return Value}\\
The function returns 0 on success and non-zero otherwise.\\

\noindent
{\bf Parameters}
\vspace{-0.1in}
\begin{description}
\item
\begin{tabular}{p{10 mm}p{145 mm}}
$agent$ & The mobile agent from which to print the code.
\end{tabular}
\end{description}

\noindent
{\bf Description}\\
This function prints the mobile agent code to the standard output.\\

\noindent
{\bf Example}\\
\noindent
%Compare with output for examples in \CPlot::\Arrow(), \CPlot::\AutoScale(),
%\CPlot::\DisplayTime(), \CPlot::\Label(), \CPlot::\TicsLabel(), 
%\CPlot::\Margins(), \CPlot::\BoundingBoxOffsets(), \CPlot::\TicsDirection(),\linebreak
%\CPlot::\TicsFormat(), and \CPlot::\Title().
%{\footnotesize\verbatiminput{template/example/Data2D.ch}}

\noindent
{\bf See Also}\\

%\CPlot::\DataThreeD(), \CPlot::\DataFile(), \CPlot::\Plotting(), \plotxy().\\
