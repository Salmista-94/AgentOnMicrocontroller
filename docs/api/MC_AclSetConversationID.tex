\rhead{\bf MC\_AclSetConversationID()}
\noindent
\vspace{5pt}
\rule{6.5in}{0.015in}
\noindent
\phantomsection
{\LARGE \bf MC\_AclSetConversationID()\index{MC\_AclSetConversationID()}}\\
\addcontentsline{toc}{section}{MC\_AclSetConversationID()}
\label{api:MC_Acl_SetConversationID()}

\noindent
{\bf Synopsis}\\
{\bf \#include $<$libmc.h$>$}\\
{\bf int MC\_AclSetConversationID}({\bf fipa\_acl\_message\_t*} acl, {\bf const char*} id);\\

\noindent
{\bf Purpose}\\
Set the conversation id on an ACL message.\\

\noindent
{\bf Return Value}\\
Returns 0 on success or non-zero on failure.\\

\noindent
{\bf Parameters}
\vspace{-0.1in}
\begin{description}
\item
\begin{tabular}{p{10 mm}p{145 mm}} 
$acl$ & An initialized ACL message. \\
$content$ & Set the conversation id field of an ACL message.
\end{tabular}
\end{description}

\noindent
{\bf Description}\\
This function sets the ``conversation-id'' field of an ACL message. 
The conversation ID is used to differentiate multiple agent conversations which
may be happening simultaneously between two agents. For more details, please
consult the FIPA specifications at \texttt{http://www.fipa.org}.\\

\noindent
{\bf See Also}\\
\texttt{
  MC\_AclSetPerformative(), MC\_AclSetSender(), MC\_AclAddReceiver(), 
    \linebreak MC\_AclAddReplyTo()
}

%\CPlot::\DataThreeD(), \CPlot::\DataFile(), \CPlot::\Plotting(), \plotxy().\\
