\rhead{\bf MC\_GetAgentXMLString()}
\noindent
\vspace{5pt}
\rule{6.5in}{0.015in}
\noindent
{\LARGE \bf MC\_GetAgentXMLString()\index{MC\_GetAgentXMLString()}}\\
\phantomsection
\addcontentsline{toc}{section}{MC\_GetAgentXMLString()}
\label{api:MC_GetAgentXMLString}

\noindent
{\bf Synopsis}\\
{\bf \#include $<$libmc.h$>$}\\
{\bf char *MC\_GetAgentXMLString}({\bf MCAgent\_t} $agent$);\\

\noindent
{\bf Purpose}\\
Retrieve a mobile agent message in XML format as a character string.\\

\noindent
{\bf Return Value}\\
The function returns an allocated character array on success and NULL on 
failure.\\

\noindent
{\bf Parameters}
\vspace{-0.1in}
\begin{description}
\item               
\begin{tabular}{p{10 mm}p{145 mm}}
$agent$ & The mobile agent from which to retrieve the XML formatted message.
\end{tabular}
\end{description}

\noindent
{\bf Description}\\
This function retrieves a mobile agent message in XML format as a character 
string. 
The return pointer is allocated by 'malloc()' and must be freed by the user.\\

\noindent
{\bf Example}\\
\noindent
{\footnotesize\verbatiminput{../demos/mobilec_c-space_functionality/cspace_misc_examples/server.c}}

\noindent
{\bf See Also}\\

%\CPlot::\DataThreeD(), \CPlot::\DataFile(), \CPlot::\Plotting(), \plotxy().\\
