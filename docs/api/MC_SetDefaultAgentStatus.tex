\rhead{\bf MC\_SetDefaultAgentStatus()}
\noindent
\vspace{5pt}
\rule{6.5in}{0.015in}
\noindent
{\LARGE \bf MC\_SetDefaultAgentStatus()\index{MC\_SetDefaultAgentStatus()}}\\
\phantomsection
\addcontentsline{toc}{section}{MC\_SetDefaultAgentStatus()}
\label{api:MC_SetDefaultAgentStatus()}

\noindent
{\bf Synopsis}\\
{\bf \#include $<$libmc.h$>$}\\
{\bf int MC\_SetDefaultAgentStatus}({\bf MCAgency\_t} $agency$, {\bf int} $status$);\\

\noindent
{\bf Purpose}\\
Set the default status of any incoming mobile agents.\\

\noindent
{\bf Return Value}\\
This function returns 0 on success and non-zero otherwise.\\

\noindent
{\bf Parameters}
\vspace{-0.1in}
\begin{description}
\item               
\begin{tabular}{p{10 mm}p{145 mm}}
$agency$ & A handle to a running agency.\\
$status$ & An integer representing the status to be assinged to any incoming 
mobile agents as their default status.
\end{tabular}
\end{description}

\noindent
{\bf Description}\\
This function is used to set the default agent status for all incoming
agents in an agency. By default, every incoming agent is set to status
``MC\_WAIT\_CH'', but that may be changed with this function.
The agent status is an enumerated type ``enum MC\_AgentStatus\_e'', which
may be seen in Table \vref{mobilec_macro}.

\noindent
{\bf Example}\\
\begin{verbatim}
MCAgency_t agency;
agency = MC_Initialize(5050, NULL);
MC_SetDefaultAgentStatus(agency, MC_AGENT_NEUTRAL);

/* etc... */
\end{verbatim}\\
\noindent
%Compare with output for examples in \CPlot::\Arrow(), \CPlot::\AutoScale(),
%\CPlot::\DisplayTime(), \CPlot::\Label(), \CPlot::\TicsLabel(), 
%\CPlot::\Margins(), \CPlot::\BoundingBoxOffsets(), \CPlot::\TicsDirection(),\linebreak
%\CPlot::\TicsFormat(), and \CPlot::\Title().
%{\footnotesize\verbatiminput{template/example/Data2D.ch}}

\noindent
{\bf See Also}\\
MC\_GetAgentStatus()

