\rhead{\bf mc\_ComposeAgentS()}
\noindent
\vspace{5pt}
\rule{6.5in}{0.015in}
\noindent
\phantomsection
{\LARGE \bf mc\_ComposeAgentS()\index{mc\_ComposeAgentS()}}\\
\addcontentsline{toc}{section}{mc\_ComposeAgentS()}

\noindent
{\bf Synopsis}\\
{\bf \#include $<$libmc.h$>$}\\
{\bf mcAgent\_t mc\_ComposeAgentS}({\bf const char*} $name$, 
                                  {\bf const char*} $home$,
                                  {\bf const char*} $owner$,
                                  {\bf const char*} $code$,
                                  {\bf const char*} $return\_var\_name$,
                                  {\bf const char*} $server$,
                                  {\bf int} $persistent$,
                                  {\bf const char*} $workgroup\_code$
																	);\\

\noindent
{\bf Purpose}\\
This function is used to compose an agent from source code.\\

\noindent
{\bf Return Value}\\
The function returns a valid agent on success and NULL otherwise.\\

\noindent
{\bf Parameters}
\vspace{-0.1in}
\begin{description}
\item
\begin{tabular}{p{30 mm}p{125 mm}} 
$name$ & The name to assign to the new agent.\\
$home$ & The home of the new agent.\\
$owner$ & The owner of the new agent.\\
$code$ & The agent C/C++ code.\\
$return\_var\_name$ & (optional) The name of the agent's return variable.\\
$server$ & The name of the destination server to send the agent.\\
$persistent$ & Whether or not the created agent should be persistent.\\
$workgroup\_code$ & (optional) The workgroup code of the agent. Only agents with matching workgroup codes are allowed to interact with each other.\\
\end{tabular}
\end{description}

\noindent
{\bf Description}\\
This function is used to create an agent C/C++ source code. 

\noindent
{\bf Example}\\
\noindent
{\footnotesize\verbatiminput{../demos/FIPA_compliant_ACL_messages/fipa_protocol/server.c}}

\noindent
{\bf See Also}\\
\texttt{mc\_ComposeAgent()}, \texttt{mc\_ComposeAgentFromFile()}

%\CPlot::\DataThreeD(), \CPlot::\DataFile(), \CPlot::\Plotting(), \plotxy().\\
